\documentclass[10pt,a4paper]{article}

% images
\usepackage[margin=1in]{geometry}
\usepackage{graphicx}
\usepackage{subfig}

% reference items
\usepackage{enumitem}

% links
\usepackage{url}
\usepackage{hyperref}

\usepackage{pdfpages}
\usepackage{pgfplots}

\usepackage{tikz}
\usetikzlibrary{arrows,automata,positioning}

\usepackage{footnote}

% maths
\usepackage{amsmath}
\usepackage{amssymb}
\usepackage{dsfont}
\usepackage{bm}

\usepackage{amsthm}

\theoremstyle{plain}
\newtheorem{theorem}{Theorem}[section]
\newtheorem{claim}[theorem]{Claim}
\newtheorem{lemma}[theorem]{Lemma}

\theoremstyle{definition}
\newtheorem{definition}[theorem]{Definition}

\newenvironment{subproof}[1][\proofname]{%
  \renewcommand{\qedsymbol}{$\blacksquare$}%
  \begin{proof}[#1]%
}{%
  \end{proof}%
}

\newtheorem*{claim*}{Claim}
\newtheorem*{corollary}{Corollary}
\newtheorem*{remark}{Remark}
\newtheorem*{fact}{Fact}

\DeclareMathOperator*{\argmax}{arg\,max}
\DeclareMathOperator*{\argmin}{arg\,min}
 
\newcommand{\code}[1]{\texttt{#1}}
\newcommand*\conj[1]{\overline{#1}}
\newcommand*\vect[1]{\bm{#1}}

% No section numbering
% \setcounter{secnumdepth}{0}

\bibliographystyle{siam}

\title{\textsc{CS908 Assignment 4 - Research Proposal}}
\author{Thomas Archbold}
\date{}

\begin{document}
\maketitle

% Introduction (Background, Aims, Problem Statement)
% Related Work (Academic Research, Existing Systems)
% Research Processes (Approach, Methodology)
% Project Management (Timeline, Constraints, Risk)
% Progress
% Conclusion

% Problem definition and literature review        /20
% Research methodology and viability              /40
% Project plan: time management and risk analysis /10
% Communication skills                            /30

\section{Introduction}

Prediction markets are exchange-traded markets\footnote{A market in which all
transactions are routed through a central source.} that trade on the outcome of
events, as opposed to traditional financial instruments. Since actors in the
market participate by putting up their own money in the form of betting on an
unknown future outcome, the market prices can indicate the beliefs held by the
market of certain events occurring. What can make these markets even more
interesting is the ability to combine these bids on events in complex ways,
giving users the freedom to make predictions over a range of different unknown
outcomes. This is the approach taken by \emph{Predictalot}~\cite{Predictalot},
which boasts a platform on which one can bet on over 9.2 quintillion outcomes
in the NCAA Men's College Basketball playoffs. This specification outlines the
plans to implement a similar combinatorial prediction market for betting on
outcomes in the Gallagher Premiership Rugby league, placing as few restrictions
as possible on the types of bets available to traders.

The rest of the specification is structured as follows. In
Section~\ref{sec:background} we outline in greater detail the motivation for
such a project, highlighting the gaps in the current literature and presenting
our specific goals for the platform. In Section~\ref{sec:relatedWork} we
discuss recent results in mechanism design for combinatorial auctions as well
as current software implementing other prediction markets. In
Section~\ref{sec:researchProcesses} we outline the methods and approaches we
plan to take in order to complete this project. Aspects related to the
effective management of the project are discussed in
Section~\ref{sec:projectManagement}. We conclude with a brief discussion of
progress made thus far in Section~\ref{sec:progress}, followed by closing
thoughts and plans for the next stages in Section~\ref{sec:conclusion}.

\section{Background}
	\label{sec:background}

	\subsection{Problem Statement}
	% combinatorial auctions hard
	% prediction markets are good for crowdsourcing public sentiment

	Prediction markets provide ways in which to bet on the occurrence of events
	in the future, and are often used to bet on a variety of circumstances --
	this could be on the outcomes of a political election, sporting events, or
	any other probabilistic event. Since there is an incentive to do well in
	such a market, by players staking their own money or units from some point
	system, people are inclined to bet how they truly feel about certain
	events, and hence public sentiment on these events can be crowdsourced to
	learn how likely they are thought to occur. Combinatorial prediction
	markets, taking inspiration from the theoretical economics and mechanism
	design literature, allows for multiple bets to be bought and sold at once,
	and it is how these are combined in complex ways that we may learn the most
	about how the public view the likelihood of separate events as being
	related.

	It is well-known, however, that computing allocations of goods to buyers
	that maximises, for example, social welfare or revenue, requires solving an
	NP-hard optimisation problem \cite{VCGNPhard}. Furthermore, for a market
	offering bets on $n$ separate events, there are $2^n$ possible ways of
	combining such bets -- how can we expect users to enter an exponential
	number of bids before they even get to participate in the market? These two
	problems are the focus of much of the literature within algorithmic
	mechanism design \cite{AMD}, a subfield within algorithmic game theory.
	Broadly, this is an area of research at the intersection of economics and
	computer science that is concerned with designing the ways in which
	self-interested agents act within a strategic environment to achieve
	certain economic properties -- truthfulness, budget balance, individual
	rationality, for example -- while ensuring that the mechanisms remain
	practical to implement. It hence makes extensive use of techniques favoured
	in computer science, most notably asymptotic analysis, randomisation, and
	approximation. The goal in the modern literature therefore departs from
	striving to compute the idealistic but impractical optimal solution,
	towards computing ones which are approximately-optimal, or ``good enough''.

	Even restricting ourselves to approximately-optimal solutions, we are faced
	with another problem -- how do we define ``good enough''? What makes, say,
	a 6-approximation for computing an allocation in a two-sided market any
	better than a 7-approximation that also achieves group-strategy
	proofness?\footnote{These are just examples to illustrate the point and may
	or may not exist.} The answer is that it depends -- trade offs must be made
	on a variety of parameters and assumptions that we make on the model. For
	example, are buyers required to submit a bid on a single bet, stopping the
	ability for bets to be combined but yielding a better approximation, or can
	they submit up to $k$ desired sets of bets, at the cost of some
	performance?
	% Since we dont know, we will play around with this

	\subsection{Aims}

\section{Related Work}
	\label{sec:relatedWork}

	\subsection{Contemporary Literature}
	\subsection{Existing systems}

\section{Research Processes}
	\label{sec:researchProcesses}

\section{Project Management}
	\label{sec:projectManagement}

\section{Progress}
	\label{sec:progress}

\section{Conclusion}
	\label{sec:conclusion}

\bibliography{bibliography}

\end{document}
